\documentclass[12pt, twoside]{article}
\usepackage[utf8]{inputenc}
\usepackage[english,russian]{babel}
\newcommand{\hdir}{.}

\usepackage{graphicx}
\usepackage{caption}
\usepackage{amssymb}
\usepackage{amsmath}
\usepackage{mathrsfs}
\usepackage{euscript}
\usepackage{upgreek}
\usepackage{array}
\usepackage{theorem}
\usepackage{graphicx}
\usepackage{subfig}
\usepackage{caption}
\usepackage{color}
\usepackage{url}

\usepackage[left=2cm,right=2cm,top=3cm,bottom=2cm,bindingoffset=0cm]{geometry}

\usepackage{fancyhdr}
\pagestyle{fancy}
\fancyhead{}
\fancyhead[LE,RO]{\thepage} 
\fancyhead[CO,CE]{Лекция 8}
\fancyhead[LO,LE]{Грабовой Андрей}

\begin{document} 

\begin{center}
{\LARGE\bf
Краска, один из рекурсивных алгоритмов
}
\end{center}

\section{Задача Castle}

\paragraph{Условие задачи.} Древний замок имеет прямоугольную форму. Замок имеет как минимум две комнаты. Пол замка можно условно поделить на $M\times N$ клеток. Каждая клетка содержит $0$ либо $1$, который задают пол и стену замку соответственно. Напишите программу, которая находит площадь наибольшей комнаты, которую можно получить при помощи удалении одной стены, то есть заменив только одну $1$ на $0$. Удалять внешние стены запрещено.
\paragraph{Технические условия.} Программа Castle считывает с устройства стандартного ввода <<план замка>>. В первой строчке имеется число $M$, во второй строке целое число $N$ --- коиличество строк и количество столбцов. $M$ следующих строк содержит по $N$  нулей либо единиц, которые идут подряд, то есть без пробелов. Первый и последний столбец, а также первый и последний столбик это внешние стены и состоят только из единиц. Программа выводит на устройство площадь максимальной комнаты, которая получится, при удалении внутренней стены.



\end{document} 