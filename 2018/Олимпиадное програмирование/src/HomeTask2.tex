\documentclass[12pt, twoside]{article}
\usepackage[utf8]{inputenc}
\usepackage[english,russian]{babel}
\newcommand{\hdir}{.}

\usepackage{graphicx}
\usepackage{caption}
\usepackage{amssymb}
\usepackage{amsmath}
\usepackage{mathrsfs}
\usepackage{euscript}
\usepackage{upgreek}
\usepackage{array}
\usepackage{theorem}
\usepackage{graphicx}
\usepackage{subfig}
\usepackage{caption}
\usepackage{color}
\usepackage{url}

\usepackage[left=2cm,right=2cm,top=3cm,bottom=2cm,bindingoffset=0cm]{geometry}

\usepackage{fancyhdr}
\pagestyle{fancy}
\fancyhead{}
\fancyhead[LE,RO]{\thepage} 
\fancyhead[CO,CE]{Домашнее задание 2}
\fancyhead[LO,LE]{Грабовой Андрей}

\begin{document} 

\begin{center}
{\LARGE\bf
Динамическое программирование 
}
\end{center}

\section{Задача Зайчик\cite{Rabbit}}

\paragraph{Условие задачи.} В нашем зоопарке появился заяц. Его поместили в клетку, и чтобы ему не было скучно, директор зоопарка распорядился поставить в его клетке лесенку. Теперь наш зайчик может прыгать по лесенке вверх, перепрыгивая через ступеньки. Лестница имеет определенное количество ступенек N. Заяц может одним прыжком преодолеть не более $К$ ступенек. Для разнообразия зайчик пытается каждый раз найти новый путь к вершине лестницы. Директору любопытно, сколько различных способов есть у зайца добраться до вершины лестницы при заданных значениях $K$ и $N$. Помогите директору написать программу, которая поможет вычислить это количество. Например, если K=3 и N=4, то существуют следующие маршруты: $1+1+1+1$, $1+1+2$, $1+2+1$, $2+1+1$, $2+2$, $1+3, 3+1$. Т.е. при данных значениях у зайца всего 7 различных маршрутов добраться до вершины лестницы.
\paragraph{Входные данные.} В единственной строке записаны два натуральных числа $K$ и $N (1 \leq K \leq N \leq 300)$. $К$ - максимальное количество ступенек, которое может преодолеть заяц одним прыжком, $N$ – общее число ступенек лестницы.
\paragraph{Выходные данные.} В единственную строку нужно вывести количество возможных вариантов различных маршрутов зайца на верхнюю ступеньку лестницы без ведущих нулей.

\begin{thebibliography}{99}
	\bibitem{Rabbit}
	\textit{Зайчик} {http://acmp.ru/index.asp?main=task\&id\_task=11}
\end{thebibliography}



\end{document} 